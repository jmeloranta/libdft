\documentclass[12pt,letterpaper]{report}
\usepackage[letterpaper,hcentering,vcentering,left=1in,top=2.45cm,right=1in,bott
om=2.45cm]{geometry}
\usepackage[latin1]{inputenc}
\usepackage{url}
\usepackage{longtable}
\usepackage{amsmath}
\usepackage{amsfonts}
\usepackage{amssymb}
\usepackage{graphicx}
\author{Jussi Eloranta}
\title{libgrid manual}
\begin{document}

\maketitle

\chapter{Prerequisites}

\section{Introduction}

Libdft provides implementation of Orsay-Trento (OT), Gross-Pitaevskii (and other 
similar) density functional for modeling superfluid $^4$He. It is built on top 
of libgrid library, which allows parallel execution on both shared memory 
CPU-based systems as well as GPUs. As libdft is a set of library functions,
each user application consists of a main C (or C++) program that makes
function calls to libdft (and possibly libgrid). 
Libdft was written by Lauri Lehtovaara, David Mateo, and Jussi Eloranta, and 
can be freely distributed according to GNU GENERAL PUBLIC LICENSE Version 3 
(see doc/GPL.txt). This project was partially supported by National Science 
Foundation grants: CHE-0949057, CHE-1262306 and DMR-1205734.

\section{Brief description of Orsay-Trento}

The following references provide the necessary background to Orsay-Trento functional(s):
\begin{itemize}
\item Original Orsay-Trento functional for $^4$He at 0 K: Phys. Rev. B 52, 1192 (1995).
\item Extension of Orsay-Trento functional to non-zero temperatures: Phys. Rev. B. 62, 17035 (2000).
\item High-density correction to Orsay-Trento: Phys. Rev. B 72, 214522 (2005).
\item Implementation of Orsay-Trento: J. Comput. Phys. 194, 78 (2004) and J. Comput. Phys. 221, 148 (2007).
\item Review article on Orsay-Trento and applications (also high-density correction to backflow): Int. Rev. Phys. Chem. 
36, 621 (2017).
\end{itemize}

% OT energy density here

% OT functional derivative here

% Note that Dupont-Roc and Gross-Pitaevskii can also be used.

\section{Installation}

Installation of libdft requires the following packages:
\begin{itemize}
\item git (a free and open source distributed version control system)
\item GNU C compiler with OpenMP support (gcc)
\item FFTW 3.x (Fast Fourier Transform package)
\item libgrid (grid library)
\end{itemize}
To install these packages on Fedora linux, use (\# implies execution with root privileges): 
\begin{verbatim}
# dnf install git gcc fftw-*
\end{verbatim}
Furthermore, libgrid must be installed before compiling libdft.

To copy the current version of libdft to subdirectory libgrid, issue 
(\% implies execution with normal user privileges):
\begin{verbatim}
% git clone https://github.com/jmeloranta/libdft.git
\end{verbatim}
To compile libdft, change to libdft source code directory and run make:
\begin{verbatim}
% cd libdft/src
% make -j
\end{verbatim}
No changes to any of the compilation options are required as it will 
automatically use the same flags as the currently installed libgrid.
Provided that the compilation completed without errors, install the library 
(as root):
\begin{verbatim}
# make install
\end{verbatim}

\chapter{Programming interface}

\section{Accessing the library routines}

To access libdft functions in C program, the following header files should be 
included:
\begin{verbatim}
#include <dft/dft.h>
#include <dft/ot.h>
\end{verbatim}

To compile and link a program using libdft and libgrid, it is most convenient 
to construct a makefile (note that the \$(CC) line has TAB as the first 
character):
\begin{verbatim}
include /usr/include/dft/make.conf

test: test.o
    $(CC) $(CFLAGS) -o test test.o $(LDFLAGS)

test.o: test.c
\end{verbatim}
This will compile the program specified in test.c and link the appropriate 
libraries automatically. Both CFLAGS and LDFLAGS are obtained automatically. 

\section{Data types}

The libdft header file ot.h defines the OT structure that contains the description of the functional to be used:

\begin{verbatim}
\end{verbatim}

\chapter{Library functions}

\chapter{Examples}


%\section{References}

\end{document}
