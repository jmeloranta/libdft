\documentclass[12pt,letterpaper]{report}
\usepackage[letterpaper,hcentering,vcentering,left=1in,top=2.45cm,right=1in,bott
om=2.45cm]{geometry}
\usepackage[latin1]{inputenc}
\usepackage{url}
\usepackage{longtable}
\usepackage{amsmath}
\usepackage{amsfonts}
\usepackage{amssymb}
\usepackage{graphicx}
\author{Jussi Eloranta}
\title{libdft manual}
\begin{document}

\maketitle

\chapter{Prerequisites}

\section{Introduction}

Libdft provides implementation of Orsay-Trento (OT), Gross-Pitaevskii (and other 
similar) density functional for modeling superfluid $^4$He. It is built on top 
of libgrid library, which allows parallel execution on both shared memory 
CPU-based systems as well as GPUs. As libdft is a set of library functions,
each user application consists of a main C (or C++) program that makes
function calls to libdft (and possibly libgrid). 
Libdft was written by Lauri Lehtovaara, David Mateo, and Jussi Eloranta, and 
can be freely distributed according to GNU GENERAL PUBLIC LICENSE Version 3 
(see doc/GPL.txt). This project was partially supported by National Science 
Foundation grants: CHE-0949057, CHE-1262306 and DMR-1205734.

\section{Brief description of Orsay-Trento}

The following references provide the necessary background to Orsay-Trento functional(s):
\begin{itemize}
\item Original Orsay-Trento functional for $^4$He at 0 K: Phys. Rev. B 52, 1192 (1995).
\item Extension of Orsay-Trento functional to non-zero temperatures: Phys. Rev. B. 62, 17035 (2000).
\item High-density correction to Orsay-Trento: Phys. Rev. B 72, 214522 (2005).
\item Implementation of Orsay-Trento: J. Comput. Phys. 194, 78 (2004) and J. Comput. Phys. 221, 148 (2007).
\item Review article on Orsay-Trento and applications (also high-density correction to backflow): Int. Rev. Phys. Chem. 36, 621 (2017).
\end{itemize}

\section{Installation}

Installation of libdft requires the following packages:
\begin{itemize}
\item git (a free and open source distributed version control system)
\item GNU C compiler with OpenMP support (gcc)
\item FFTW 3.x (Fast Fourier Transform package)
\item libgrid (grid library)
\end{itemize}
To install these packages on Fedora linux, use (\# implies execution with root privileges): 
\begin{verbatim}
# dnf install git gcc fftw-*
\end{verbatim}
Furthermore, libgrid must be installed before compiling libdft.

To copy the current version of libdft to subdirectory libgrid, issue 
(\% implies execution with normal user privileges):
\begin{verbatim}
% git clone https://github.com/jmeloranta/libdft.git
\end{verbatim}
To compile libdft, change to libdft source code directory and run make:
\begin{verbatim}
% cd libdft/src
% make -j
\end{verbatim}
No changes to any of the compilation options are required as it will 
automatically use the same flags as the currently installed libgrid.
Provided that the compilation completed without errors, install the library 
(as root):
\begin{verbatim}
# make install
\end{verbatim}

\chapter{Programming interface}

\section{Accessing the library routines}

To access libdft functions in C program, the following header files should be 
included:
\begin{verbatim}
#include <dft/dft.h>
#include <dft/ot.h>
\end{verbatim}

To compile and link a program using libdft and libgrid, it is most convenient 
to construct a makefile (note that the \$(CC) line has TAB as the first 
character):
\begin{verbatim}
include /usr/include/dft/make.conf

test: test.o
    $(CC) $(CFLAGS) -o test test.o $(LDFLAGS)

test.o: test.c
\end{verbatim}
This will compile the program specified in test.c and link the appropriate 
libraries automatically. Both CFLAGS and LDFLAGS are obtained automatically. Note that if you changed the ROOT directory when installing libgrid, you need to replace /usr above with that directory.

\section{Data types}

The libdft header file ot.h defines the OT structure (data type dft\_ot\_functional) that contains the description of the functional to be used:

\begin{verbatim}
typedef struct dft_ot_functional_struct { /* All values in atomic units */
  INT model;              /* Functional DFT_OT_* (Orsay-Trento), DFT_DR (Dupont-Roc) */        
                          /* DFT_GP (Gross-Pitaevskii) */
  REAL b;                 /* Lennard-Jones integral value for bulk (not used in */
                          /* functional) */
  REAL c2;                /* Orsay-Trento short-range correlation parameter c_2 */
                          /* (2nd power) */
  REAL c2_exp;            /* Exponent for the above (= 2 for OT) */
  REAL c3;                /* Orsay-Trento short-range correlation parameter c_3 */
                          /* (3rd power) */
  REAL c3_exp;            /* Exponent for the above (= 3 for OT) */
  REAL c4;                /* Thermal Orsay-Trento parameter c_4 (Ancilotto et al.) */
  REAL rho_0s;            /* Orsay-Trento kinetic energy correlation parameter */
                          /* \rho_{0s} */
  REAL alpha_s;           /* Orsay-Trento kinetic energy correlation parameter */
                          /* \alpha_s */
  REAL l_g;               /* Width of gaussian F used in kinetic correlation */
  REAL mass;              /* ^4He mass */
  REAL rho0;              /* Uniform bulk liquid density at saturated vapor pressure */
  REAL temp;              /* Temperature (Kelvin) */
  dft_common_lj lj_params;/* Lennard-Jones parameters */
  dft_ot_bf bf_params;    /* Backflow functional parameters */
  rgrid *lennard_jones;   /* Grid holding Fourier transformed effective Lennard-Jones*/
                          /* function */
  rgrid *spherical_avg;   /* Grid holding Fourier transformed spherical average */
                          /* function */
  rgrid *gaussian_tf;     /* Grid holding Fourier transformed gaussian F */
                          /* (kinetic correlation) */ 
  rgrid *gaussian_x_tf;   /* Grid holding Fourier transformed derivative of */
                          /* gaussian F (dF/dx; kinetic correlation) */ 
  rgrid *gaussian_y_tf;   /* Grid holding Fourier transformed derivative of */
                          /* gaussian F (dF/dy; kinetic correlation) */ 
  rgrid *gaussian_z_tf;   /* Grid holding Fourier transformed derivative of */
                          /* gaussian F (dF/dz; kinetic correlation) */ 
  rgrid *backflow_pot;    /* Grid holding Fourier transformed bacflow function (V_j) */
  REAL beta;              /* High density correction parameter \beta */
  REAL rhom;              /* High density correction parameter \rho_m */
  REAL C;                 /* High density correction parameter C */
  REAL mu0;               /* Determines Gross-Pitaevskii contact strength: */
                          /* \mu_0 / \rho_0 */
  REAL xi;                /* High density correction parameter for backflow \xi */
  REAL rhobf;             /* High density correction parameter for backflow \rho_{bf}*/
  REAL div_epsilon;       /* Epsilon to use when dividing by density */
                          /* (affects, e.g., backflow) */
  REAL max_veloc;         /* Maximum velocity value allowed for BF */
  REAL max_bfpot;         /* Maximum allowed potential value for BF */
  rgrid *workspace1;      /* Workspace 1 (these may be NULL if not allocated) */
  rgrid *workspace1;      /* Workspace 1 (these may be NULL if not allocated) */
  rgrid *workspace2;      /* Workspace 2 */
  rgrid *workspace3;      /* Workspace 3 */
  rgrid *workspace4;      /* Workspace 4 */
  rgrid *workspace5;      /* Workspace 5 */
  rgrid *workspace6;      /* Workspace 6 */
  rgrid *workspace7;      /* Workspace 7 */
  rgrid *workspace8;      /* Workspace 8 */
  rgrid *workspace9;      /* Workspace 9 */
  rgrid *density;         /* Liquid density */
} dft_ot_functional;
\end{verbatim}

\noindent
These fields are initialized by calling function dft\_ot\_alloc() (see below). The density and workspace grids are used during evaluation of the Orsay-Trento potential, but they may be used for other purposes outside dft\_ot\_potential(). At present the workspace allocation is as follows (density is used by all functionals):

\begin{tabular}{lll}
Functional name & libdft notation & Number of workspaces used\\
\cline{1-3}
Gross-Pitaevskii & DFT\_GP & Workspace 1\\
Plain Orsay-Trento & DFT\_OT\_PLAIN & Workspaces 1 - 3.\\
 & & Includes also thermal DFT (DFT\_OT\_T*).\\
O-T with KC & DFT\_OT\_KC & Workspaces 1 - 6.\\
O-T with BF & DFT\_OT\_BACKFLOW & Workspaces 1 - 9.\\
\end{tabular}

\noindent
The functionals listed in the above table are specified in the next section.

\section{Functionals and their modifiers}

The following functionals and their modifiers have been implemented in libdft:

\begin{longtable}{p{.33\textwidth} p{.4\textwidth} p{.33\textwidth}}
Functional & libdft notation for the functional & Class\\
\cline{1-3}
Plain Orsay-Trento & DFT\_PLAIN & Functional\\
Dupont-Roc & DFT\_DR & Functional\\
Gross-Pitaevskii & DFT\_GP & Functional\\
Thermal O-T at 0 K & DFT\_OT\_T0MK & Functional\\
Thermal O-T at 0.4 K & DFT\_OT\_T400MK & Functional\\
Thermal O-T at 0.6 K & DFT\_OT\_T600MK & Functional\\
Thermal O-T at 0.8 K & DFT\_OT\_T800MK & Functional\\
Thermal O-T at 1.2 K & DFT\_OT\_T1200MK & Functional\\
Thermal O-T at 1.4 K & DFT\_OT\_T1400MK & Functional\\
Thermal O-T at 1.6 K & DFT\_OT\_T1600MK & Functional\\
Thermal O-T at 1.8 K & DFT\_OT\_T1800MK & Functional\\
Thermal O-T at 2.0 K & DFT\_OT\_T2000MK & Functional\\
Thermal O-T at 2.1 K & DFT\_OT\_T2100MK & Functional\\
Thermal O-T at 2.2 K & DFT\_OT\_T2200MK & Functional\\
Thermal O-T at 2.4 K & DFT\_OT\_T2400MK & Functional\\
Thermal O-T at 2.6 K & DFT\_OT\_T2600MK & Functional\\
Thermal O-T at 2.8 K & DFT\_OT\_T2800MK & Functional\\
Thermal O-T at 3.0 K & DFT\_OT\_T3000MK & Functional\\
O-T high density correction1 & DFT\_OT\_HD & Modifier for DFT\_OT\_PLAIN\\
O-T high density correction2 & DFT\_OT\_HD2 & Modifier for DFT\_OT\_PLAIN\\
O-T backflow & DFT\_OT\_BACKFLOW & Modifier to include the backflow term in Orsay-Trento\\
O-T kinetic correlation & DFT\_OT\_KC & Modifier to include the kinetic energy correlation term in Orsay-Trento.\\ 
\end{longtable}
\noindent
The two high-density corrections (1 and 2) refer to two slightly different parametrizations of the penalty term. Invoking either of these two modifiers will also include the high-density correction to the backflow functional.

To apply a functional and the desired modifiers, use logical or. For example, to use the full Orsay-Trento, specify DFT\_OT\_PLAIN $|$ DFT\_OT\_KC $|$ DFT\_OT\_BACKFLOW. Here $|$ is the or operator in C (and operator would be \&).

Libdft include files also define the following useful constants:

\begin{tabular}{ll}
Name & Description\\
\cline{1-2}
DFT\_KB & Boltzmann constant in atomic units.\\
DFT\_HELIUM\_MASS & $^4$He atom mass.\\
DFT\_MIN\_SUBSTEPS & Minimum \# of steps for grid\_smooth\_map() function to be used.\\
DFT\_MAX\_SUBSTEPS & Maximum \# of steps for grid\_smooth\_map() function to be used.\\
\end{tabular}\\

\noindent
The latter two constants are often used when calling dft\_ot\_alloc(). Just like libgrid, libdft uses atomic units everywhere.

\chapter{Library functions}

Libdft provides only the routines that are specific to Orsay-Trento (Gross-Pitaevskii etc.) models. All common grid operations are handled by direct calls to libgrid. Each available library routine in libdft is described below.

\section{Classical DFT routines}

\input{classical-functions}

\section{Common routines}

\input{common-functions}

\section{Bulk helium routines (experimental)}

\input{helium-bulk-exp-functions}

\section{Bulk helium routines (Orsay-Trento)}

\input{helium-bulk-ot-functions}

\section{Initial guess routines}

\input{helium-initial-guess}

\section{Orsay-Trento (helium)}

\input{helium-orsay-trento}

\section{Spectroscopy}

\input{spectroscopy}

\chapter{Examples}

\subsection{Orsay-Trento: 1-D ``bubble"}

Implement a wall (1-D symmetry) in bulk superfluid helium. Move the wall for a short time and then follow the dynamics. This emits both bright and dark solitons around the wall.

\begin{verbatim}
/*
 * "One dimensional bubble" propagating in superfluid helium (propagating along Z).
 * 1-D version with X & Y coordinates integrated over in the non-linear
 * potential.
 *
 * All input in a.u. except the time step, which is fs.
 *
 */

/* Required system headers */
#include <stdlib.h>
#include <stdio.h>
#include <math.h>

/* libgrid headers */
#include <grid/grid.h>
#include <grid/au.h>

/* libdft headers */
#include <dft/dft.h>
#include <dft/ot.h>

#define TS 1.0 /* Time step (fs) */
#define NZ (32768)  /* Length of the 1-D grid */
#define STEP 0.2    /* Step length for the grid */
#define IITER 200000 /* Number of warm-up imaginary time iterations */
#define SITER 250000 /* Stop liquid flow after this many iterations */
#define MAXITER 80000000  /* Maximum iterations */
#define NTH 2000          /* Output liquid density every NTH iterations */
#define VZ (2.0 / GRID_AUTOMPS)  /* Liquid velocity (m/s) */

#define PRESSURE (0.0 / GRID_AUTOBAR)  /* External pressure (bar) */
#define THREADS 16                     /* Use this many OpenMP threads */

/* Use Predict-correct for propagation? */
#define PC

/* Bubble parameters - exponential repulsion (approx. electron bubble) - RADD = 19.0 */
#define A0 (3.8003E5 / GRID_AUTOK)
#define A1 (1.6245 * GRID_AUTOANG)
#define A2 0.0
#define A3 0.0
#define A4 0.0
#define A5 0.0
#define RMIN 2.0
#define RADD 6.0

/* Initial guess for bubble radius (imag. time) */
#define BUBBLE_RADIUS (15.0 / GRID_AUTOANG)

/* Function generating the initial guess (1-d sphere) */
REAL complex bubble_init(void *prm, REAL x, REAL y, REAL z) {

  double *rho0 = (REAL *) prm;

  if(FABS(z) < BUBBLE_RADIUS) return 0.0;
  return SQRT(*rho0);
}

/* Round velocity to fit the simulation box */
REAL round_veloc(REAL veloc) {

  INT n;
  REAL v;

  n = (INT) (0.5 + (NZ * STEP * DFT_HELIUM_MASS * veloc) / (HBAR * 2.0 * M_PI));
  v = ((REAL) n) * HBAR * 2.0 * M_PI / (NZ * STEP * DFT_HELIUM_MASS);
  fprintf(stderr, "Requested velocity = %le m/s.\n", veloc * GRID_AUTOMPS);
  fprintf(stderr, "Nearest velocity compatible with PBC = %le m/s.\n", v * GRID_AUTOMPS);
  return v;
}

/* Given liquid velocity, calculate the momentum */
REAL momentum(REAL vz) {

  return DFT_HELIUM_MASS * vz / HBAR;
}

/* Potential producing the bubble (centered at origin, z = 0) */
REAL bubble(void *asd, REAL x, REAL y, REAL z) {

  REAL r, r2, r4, r6, r8, r10;

  r = FABS(z);
  r -= RADD;
  if(r < RMIN) r = RMIN;

  r2 = r * r;
  r4 = r2 * r2;
  r6 = r4 * r2;
  r8 = r6 * r2;
  r10 = r8 * r2;
  /* Exponential repulsion + dispersion series */
  return A0 * EXP(-A1 * r) - A2 / r4 - A3 / r6 - A4 / r8 - A5 / r10;
}

/* Main program - we start execution here */
int main(int argc, char **argv) {

  rgrid *density, *ext_pot; /* Real grids for density and external potential */
  cgrid *potential_store;   /* Complex grid holding potential */
  wf *gwf, *gwfp;           /* Wave function + predicted wave function */
  dft_ot_functional *otf;   /* Functional pointer to define DFT */
  INT iter;                 /* Iteration counter */
  REAL rho0, mu0, vz, kz;   /* liquid density, chemical potential, velocity, momentum */
  char buf[512];            /* Buffer for file name */
  REAL complex tstep;       /* Time step (complex) */
  grid_timer timer;         /* Timer structure to record execution time */

#ifdef USE_CUDA
  cuda_enable(1);           /* If cuda available, enable it */
#endif

  /* Initialize threads & use wisdom */
  grid_set_fftw_flags(1);    /* FFTW planning = FFTW_MEASURE */
  grid_threads_init(THREADS);/* Initialize OpenMP threads */
  grid_fft_read_wisdom(NULL);/* Use FFTW wisdom if available */

  /* Allocate wave functions: periodic boundaries and 2nd order FFT propagator */
  if(!(gwf = grid_wf_alloc(1, 1, NZ, STEP, DFT_HELIUM_MASS, WF_PERIODIC_BOUNDARY,
                           WF_2ND_ORDER_FFT, "gwf"))) {
    fprintf(stderr, "Cannot allocate gwf.\n");
    exit(1);
  }
  gwfp = grid_wf_clone(gwf, "gwfp"); /* Clone gwf to gwfp (same but new grid) */

  /* Moving background at velocity VZ */
  vz = round_veloc(VZ);
  printf("VZ = " FMT_R " m/s\n", vz * GRID_AUTOMPS);
  kz = momentum(VZ);
  /* Set the moving background momentum to wave functions */
  cgrid_set_momentum(gwf->grid, 0.0, 0.0, kz);
  cgrid_set_momentum(gwfp->grid, 0.0, 0.0, kz);

  /* Allocate OT functional (full Orsay-Trento) */
  if(!(otf = dft_ot_alloc(DFT_OT_PLAIN | DFT_OT_BACKFLOW | DFT_OT_KC, gwf,
                          DFT_MIN_SUBSTEPS, DFT_MAX_SUBSTEPS))) {
    fprintf(stderr, "Cannot allocate otf.\n");
    exit(1);
  }

  /* Bulk density at pressure PRESSURE */
  rho0 = dft_ot_bulk_density_pressurized(otf, PRESSURE);
  /* Chemical potential at pressure PRESSURE + hbar^2 kz * kz / (2 * mass) */
  /* So, mu0 = mu0 + moving background contribution */
  mu0 = dft_ot_bulk_chempot_pressurized(otf, PRESSURE) + (HBAR * HBAR / 
                                       (2.0 * gwf->mass)) * kz * kz;
  printf("mu0 = " FMT_R " K/atom, rho0 = " FMT_R " Angs^-3.\n", mu0 * GRID_AUTOK, 
         rho0 / (GRID_AUTOANG * GRID_AUTOANG * GRID_AUTOANG));

  /* Use the real grid in otf structure (otf->density) rather than allocate new grid */
  density = otf->density;
  /* Allocate space for potential grid */
  potential_store = cgrid_clone(gwf->grid, "Potential store");
  /* Allocate space for external potential */
  ext_pot = rgrid_clone(density, "ext_pot");

  /* set up external potential (function bubble) */
  rgrid_map(ext_pot, bubble, NULL);

  /* set up initial density */
  if(argc == 2) {
    /* Read starting point from checkpoint file */
    FILE *fp;
    if(!(fp = fopen(argv[1], "r"))) {
      fprintf(stderr, "Can't open checkpoint .grd file.\n");
      exit(1);
    }
    sscanf(argv[1], "bubble-" FMT_I ".grd", &iter);
    cgrid_read(gwf->grid, fp);
    fclose(fp);
    fprintf(stderr, "Check point from %s with iteration = " FMT_I "\n", argv[1], iter);
  } else {
    /* Bubble initial guess */
    cgrid_map(gwf->grid, bubble_init, &rho0);
    iter = 0;
  }

  /* Main loop over iterations */
  for ( ; iter < MAXITER; iter++) {

    /* Output every NTH iteration */
    if(!(iter % NTH)) {
      sprintf(buf, "bubble-" FMT_I, iter); /* construct file name */
      grid_wf_density(gwf, density);  /* get density from gwf */
      rgrid_write_grid(buf, density); /* write density to disk */
    }

    /* determine time step */
    if(iter < IITER) tstep = -I * TS; /* Imaginary time */
    else tstep = TS; /* Real time */

    /* AFter SITER's, stop the flow */
    if(iter > SITER) {
      /* Reset background velocity to zero */
      cgrid_set_momentum(gwf->grid, 0.0, 0.0, 0.0);
      cgrid_set_momentum(gwfp->grid, 0.0, 0.0, 0.0);
      /* Reset the chemical potential (remove moving background contribution) */
      mu0 = dft_ot_bulk_chempot_pressurized(otf, PRESSURE);
    }

    grid_timer_start(&timer); /* start iteration timer */
    cgrid_zero(potential_store);  /* clear potential */
    /* If PC is defined, use predict-correct */
    /* If not, use single stepping */
#ifdef PC
    /* predict-correct */
    /* Add O-T potential at current time */
    dft_ot_potential(otf, potential_store, gwf);
    /* Add external potential */
    grid_add_real_to_complex_re(potential_store, ext_pot);
     /* Add -chemical potential */ 
    cgrid_add(potential_store, -mu0);
     /* predict step */
    grid_wf_propagate_predict(gwf, gwfp, potential_store, tstep / GRID_AUTOFS);
    /* Get O-T potential at prediction point */
    dft_ot_potential(otf, potential_store, gwfp);
    /* Add external potential */
    grid_add_real_to_complex_re(potential_store, ext_pot);
    /* add -chemical potential */
    cgrid_add(potential_store, -mu0);    
    /* For correct step, use potential (current + future) / 2 */          
    cgrid_multiply(potential_store, 0.5);
    /* Take the correct step */
    grid_wf_propagate_correct(gwf, potential_store, tstep / GRID_AUTOFS); 
#else
    /* single stepping */
    /* Get O-T potential */
    dft_ot_potential(otf, potential_store, gwf);
    /* Add external potential */
    grid_add_real_to_complex_re(potential_store, ext_pot);
    /* Add -chemical potential */
    cgrid_add(potential_store, -mu0);
    /* Propagate */             
    grid_wf_propagate(gwf, potential_store, tstep / GRID_AUTOFS);  
#endif

    /* Report Wall time used for the current iteration */
    printf("Iteration " FMT_I " - Wall clock time = " FMT_R " seconds.\n", iter,
           grid_timer_wall_clock_time(&timer));
    /* After five iterations, write out FFTW wisdom */
    if(iter == 5) grid_fft_write_wisdom(NULL);
  }
  return 0;  /* The End */
}
\end{verbatim}

\end{document}
